% Created 2020-10-26 Mon 14:23
% Intended LaTeX compiler: pdflatex
\documentclass[11pt]{article}
\usepackage[utf8]{inputenc}
\usepackage[T1]{fontenc}
\usepackage{graphicx}
\usepackage{grffile}
\usepackage{longtable}
\usepackage{wrapfig}
\usepackage{rotating}
\usepackage[normalem]{ulem}
\usepackage{amsmath}
\usepackage{textcomp}
\usepackage{amssymb}
\usepackage{capt-of}
\usepackage{hyperref}
\usepackage[portuguese]{babel}
\usepackage{mathtools}
\usepackage[binary-units=true]{siunitx}
\usepackage[top=0.5cm,bottom=1.5cm,left=2cm,right=2cm]{geometry}
\usepackage{mdframed}
\usepackage{listings}
\usepackage{algpseudocode}
\usepackage[Algoritmo]{algorithm}
\usepackage{tikz}
\usepackage{xcolor}
\usepackage{colortbl}
\usepackage{graphicx,wrapfig,lipsum}
\usepackage{pifont}
\usepackage{subfigure}
\usepackage{rotating}
\usepackage{multirow}
\usepackage{tablefootnote}
\usepackage{enumitem}
\usepackage{natbib}
\usepackage{dblfloatfix}
\usepackage{color, colortbl}
\usepackage{chngcntr}
\usepackage{epstopdf}
\usepackage{comment}
\usepackage{float}
\author{Heitor Lourenço Werneck}
\date{}
\title{Heurística e Metaheurísticas\\\medskip
\large Atividade Avaliativa 3}
\hypersetup{
 pdfauthor={Heitor Lourenço Werneck},
 pdftitle={Heurística e Metaheurísticas},
 pdfkeywords={},
 pdfsubject={},
 pdfcreator={Emacs 27.1 (Org mode 9.3)}, 
 pdflang={Portuguese}}
\begin{document}

\maketitle

\section{Introdução}
\label{sec:org1842bc3}

O GRASP é um método que utiliza como princípio a combinação de um método construtivo com busca local, em um procedimento iterativo com iterações completamente independentes.

Já o Path Relinking é um método que faz um balanço entre intensificação e diversificação, o método considera um par de soluções e o objetivo e chegar na solução guia a partir da solução de partida, por meio disto ele consegue os feitos citados anteriormente.

A combinação dos dois métodos será estudada a seguir.

\section{Estrátegias e implementação}
\label{sec:orgc542255}

Para a formulação do GRASP primeiro foi utilizado uma heuristica construtiva por valor, ou seja, a lista restrita de candidatos é \(\{j | c_j \geq c_{max} - \alpha(c_{max}-c_{min})\}\). O alpha utilizado foi \(0.7\), isso tendo uma lista bem diversa em soluções.

O algoritmo de busca local utilizado foi o VND(Variable Neighborhood Descent), sendo este algoritmo uma busca local que utiliza vizinhanças sucessivas na descida até um ótimo local. Se uma solução \(s\) não é melhorada na sua vizinhança atual/corrente \(N^k(s)\), a estrutura é alterada da vizinhança \(N^k\) para \(N^{k+1}\). Se a solução é melhorada então a busca se inicializa de novo na nova solução melhorada na primeira vizinhança.

Para a implementação do VND foram utilizadas duas vizinhanças, uma \(N^1\) e outra \(N^2\). A vizinhança \(N^1\) troca um bit da representação por vez e testa a solução com o bit trocado e guarda a melhor solução com 1 bit trocado. A vizinhança \(N^2\) troca dois bits da representação por vez e testa a solução com os bits trocados e guarda a melhor solução.
\end{document}